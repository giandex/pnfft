%------------------------------------------------------------------------------
\section{Introduction}
%------------------------------------------------------------------------------

\begin{compactitem}
  \item NFFT consists of 3 steps: ...
  \item describe the data decomposition
  \item make clear that 3d-decomposition is possible but 2d-decomposition is more natural
\end{compactitem}
\begin{compactitem}
  \item Details on the algorithms can be found in \cite{PiPo12}.
  \item Details on the underlying parallel FFT can be found in \cite{Pi12}.
\end{compactitem}

local NFFT specific variables
\begin{compactitem}
  \item \verb+ptrdiff_t local_M+
  \item \verb+ptrdiff_t local_N[3]+, \verb+ptrdiff_t local_N_start[3]+
  \item \verb+double lower_border[3]+, \verb+double upper_border[3]+
\end{compactitem}

global NFFT specific variables
\begin{compactitem}
  \item \verb+ptrdiff_t N[3]+
  \item \verb+ptrdiff_t n[3]+
  \item \verb+int m+
  \item \verb+double x_max[3]+
\end{compactitem}



2d data decomposition with non-transposed Fourier coefficients
\begin{equation*}
  \hat N_0 / P_0 \times \hat N_1 / P_1 \times \hat N_2
  \nfftarrow
  C_0 / P_0 \times C_1 / P_1 \times C_2
\end{equation*}

2d data decomposition with transposed Fourier coefficients
\begin{equation*}
  \hat N_1 / P_0 \times \hat N_2 / P_1 \times \hat N_0
  \nfftarrow
  C_0 / P_0 \times C_1 / P_1 \times C_2
\end{equation*}

3d data decomposition with non-transposed Fourier coefficients
\begin{equation*}
  \hat N_0 / P_0 \times \hat N_1 / P_1 \times \hat N_2 / P_2
  \nfftarrow
  C_0 / P_0 \times C_1 / P_1 \times C_2 / P_2
\end{equation*}

3d data decomposition with transposed Fourier coefficients with $P_2 = Q_0 Q_1$
\begin{equation*}
  \hat N_1 / (P_0 Q_0) \times \hat N_2 / (P_1 Q_1) \times \hat N_0
  \nfftarrow
  C_0 / P_0 \times C_1 / P_1 \times C_2 / P_2
\end{equation*}


\subsection{Parallel NFFT Workflow}

\begin{compactitem}
  \item create a simple test program and describe it here
\end{compactitem}


\begin{compactitem}
  \item get block distribution of Fourier coefficients and nodes
  \item call PNFFT planner
  \item init Fourier coefficients and nodes
  \item precomputations that depend on the nodes
  \item execute PNFFT plan
  \item read results
  \item finalize PNFFT
\end{compactitem}




%------------------------------------------------------------------------------
\section{Special Features}
%------------------------------------------------------------------------------


\begin{compactitem}
  \item flags \code{PNFFT_OMIT_DECONV}, \code{PNFFT_OMIT_FFT}, \code{PNFFT_OMIT_CONV}
  \item \code{pnfft_trafo} and \code{pnfft_adj} can be called with \code{nodes == NULL} if and only if \code{PNFFT_OMIT_CONV} is set. 
    Direct computation with \code{nodes == NULL} does not make sence. Therefore, it immediately returns without doing anything.
    Passing only some nodes as NULL pointer may lead to deadlocks. I.e., in parallel calls always pass all or none nodes as NULL.
  \item \code{PNFFT_COMPUTE_F}, \code{PNFFT_COMPUTE_GRAD_F}, \code{PNFFT_HESSIAN_F} must be set in planning step in order to compute the correct FFT-values.
        This makes a difference in the number of FFTs for ik-diff but not for ad-diff. However, also for ad-diff we need the info in order to optimize precomputation
        of fct. value and derivatives (TODO: do we always precompute fct. values, 1st order and 2nd derivative, or do we check the flags?)
        They can also be changed on node-level?
  \item \code{PNFFT_COMPUTE_ACCUMULATED} with pnfft_adj makes that the Fourier coefficients are not zeroed at the beginning.
    I.e., multiple calls of adjoint interpolation accumulate in the Fourier coefficients. Typical use case is interpolation for multiple sets of nodes (charges/dipoles),
    without immediate computation of FFT (use \code{PNFFT_OMIT_FFT} and \code{PNFFT_OMIT_DECONV}). After several convolutions we compute the FFT and deconvolution
    on the accumulated values.
  \item \code{PNFFT_COMPUTE_ACCUMULATED} with pnfft_trafo makes that the fct. values, gradients and Hessian are not zeroed at the beginning of trafo.
    Use case: compute deconv and FFT in one call. Compute fct. values, gradients, Hessian for same set of nodes in separate calls. TODO: Does this flag make sense with trafo?
  \item Question: Is \code{PNFFT_OMIT_CONV} equivalent to not set any of these flags \code{PNFFT_COMPUTE_F}, \code{PNFFT_COMPUTE_GRAD_F}, \code{PNFFT_COMPUTE_HEASSIAN_F}?
    Answer: No! These 3 flags are also needed for the correct FFT execution, e.g. with ik-diff we need to call/skip FFTs for each derivative.
    But: Is this part of planning or execution? Part of both. 
      Part of planning because of memory allocation. 
      Part of execution because FFTs are expensive and we only call them if necesarry.
\end{compactitem}


\subsection{Transposed Fourier coefficients}
A parallel transpose FFT algorithm typically ends up with a transposed order of the output array.
Start with


Similar to PFFT, our parallel NFFT supports an optimization flag that disables the backward transpositions. Therefore, one must work on a transposed array of Fourier coefficients.


\subsection{Truncated Torus}
PNFFT support the special case, where the nodes $\mathbf x_j$ fulfill the restriction $\mathbf x_j \in \left[-\frac{C_0}{2},\frac{C_0}{2} \right]$

Copy from PNFFT paper:\\
In addition, we pay special attention to the case where all the nonequispaced nodes $\mathbf x_j$
are contained in a special subset of the torus $\T^3$. For $\mathbf C=(C_0,C_1,C_2)^\top\in\R^3$ with $0<C_0,C_1,C_2\le 1$
we define the truncated torus $\T^3_{\mathbf C} := [-\frac{C_0}{2},\frac{C_0}{2})\times [-\frac{C_1}{2},\frac{C_1}{2}) \times [-\frac{C_2}{2},\frac{C_2}{2})$.
For the parallel NFFT we assume $\mathbf x_j\in\T^3_{\mathbf C}$ for every $j=1,\hdots,M$.
Obviously, for $C_0=C_1=C_2=1$ this corresponds to the serial NFFT, where the nodes $\mathbf x_j$ are contained in the whole three-dimensional torus $\T^3$.
This slight generalization is necessary in order to assure a load balanced distribution of nodes $\mathbf x_j$ whenever the nodes are
concentrated in the center of the box.


%------------------------------------------------------------------------------
\section{Download and Install}
%------------------------------------------------------------------------------
\begin{compactitem}
  \item download, configure, make
  \item Advice for developers: install new version of autotools (add the script here)
\end{compactitem}





